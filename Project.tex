\documentclass{article}
\usepackage{hyperref}
\usepackage{listings}

\author{
	Dube, Fletcher\\
	\texttt{\#921441405}
	\and
	Redivo, Leonardo\\
	\texttt{\#918072175}
	\and
	Voragen, Richard\\
	\texttt{\#917981018}
}
\title{Parallel Mandelbrot: \\An exercise in CUDA and OpenMP}

\begin{document}
\maketitle

\section{Introduction}
The \emph{Mandelbrot set} is the set of complex numbers $c$ for which the function $$f_c(z) = z^2 + c$$ does not diverge to infinity when iterated from $z = 0$ \cite{mandel_wiki}. In simple terms: choose a complex number $c$, and let $z = 0$. Compute the result, and let $z$ be that result. Continue iterating until the function results in a value that exceeds some threshold, or until some large number of iterations have been computed. If the function exceeds the threshold, the function diverges and $c$ is \emph{not} in the set. Otherwise, the function does not diverge and $c$ is included in the set. In our program, each value for $c$ which diverges is color coded according to how many iterations run before the threshold is exceeded.

\section{OpenMP}

\section{CUDA}
There are a number of 

\appendix
\section{Code}
\lstset{language=C,stringstyle=\ttfamily, showstringspaces=false, numbers=left, frame=single, framexrightmargin=0pt, columns=fullflexible, breaklines=true, breakatwhitespace=true}
\lstinputlisting{mandelbrot.cu}

\section{``Who did what?''}

\bibliography{biblio} 
\bibliographystyle{ieeetr}

\end{document}